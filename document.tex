%%This is a very basic article template.
%%There is just one section and two subsections.

\documentclass[accentcolor=tud9c,10pt,nochapname]{tudexercise}
\usepackage{mathtools}
\DeclareMathSizes{10}{12}{8}{8}
\begin{document}





\title{Stochastische Signale und Systeme}
\subtitle{Zusammenfassung Formeln}
\author{Daniel Thiem}
\maketitle

\tableofcontents
\numberwithin{equation}{section}
\section{Kombinatorik \& reine Stochastik}

\subsection{Wahrscheinlichkeitsdichtefunktion}
\begin{equation}
	f(x) = P(X=x)
\end{equation}

\subsection{Verteilungsfunktion}
\begin{equation}
	F(x) = P(X\leq x) = \int\limits_{-\infty}^x f(t)dt 
\end{equation}

\subsection{Formel von Bayes}

\begin{equation}
	P(A|B) = \frac{P(A \cap B)}{P(B)} 
	\Rightarrow P(A_k|B) = \frac{P(A_k \cdot P(B|A_k))}{\sum\limits_{i=1}^n P(B|A_i) \cdot P(A_i)}
\end{equation}



\subsection{Erwartungswerte}
\subsubsection{Allgmeine Erwartungswertberechnung}


Sei $f(x)$ die Wahrscheinlichkeitsdichtefunktion von $X$

\begin{equation}
	E(X) = \int\limits_{-\infty}^\infty x \cdot f(x) dx
\end{equation}

\subsubsection{Erweiterte Erwartungswertberechnung}

Sei $Y=g(X)$ und $f(x)$ die Wahrscheinlichkeitsdichtefunktion von $X$

\begin{equation}
	E[Y] = E[g(X)] =  \int\limits_{-\infty}^\infty g(x) \cdot f(x) dx
\end{equation}

\subsubsection{Rechenregeln f�r Erwartungswerte}
\begin{equation}
	E[A \cdot B] = E[A] \cdot E[B]
\end{equation}4

\begin{equation}
	E[aX +b] = aE[X] + b
\end{equation}

\subsection{Verteilungen}

\subsubsection{Normalverteilung}
\begin{equation}
f(x) = \frac{1}{\sigma \sqrt{2 \pi}} e^{- \frac{1}{2} \left(\frac{t-\mu}{\sigma}\right)^2}
\end{equation}


\section{Prozesse}

\subsection{Strikte Stationarit�t}
\begin{equation}
F_x(x_1,\dots ,x_N;n_1,\dots,n_N) = F_x(x_1,\dots ,x_N;n_1+n_0,\dots ,n_N+n_0) \quad \text{mit $N\rightarrow \infty$}
\end{equation}

\subsection{Second order moment function(SOMF)}
\begin{equation}
r_{XX}(n_1,n_2)=E[X(n_1)X(n_2)]
\end{equation}
\subsubsection{Station�r im weiteren Sinne}
\begin{subequations}
\begin{align}
E[X(n)]&=\text{const.} \\
r_{XX}(n_1,n_2) &= r_{XX}(\kappa) = E[X(n+\kappa)\cdot X(n)] \quad \text{mit} \quad \kappa = |n_2-n_1|
\end{align}
\end{subequations}
\subsubsection{Eigenschaften der SOMF}
\begin{subequations}
\begin{align}
r_{XX}(0) &= E[X(n)^2]=\sigma_X^2+\mu_x^2 \\
r_{XX}(\kappa) &= r_{XX}(-\kappa) \\
r_{XX}(0) &\geq|r_{XX}(\kappa)| \quad ,|\kappa|>0
\end{align}
\end{subequations}

\subsection{Cross-SOMF}
\begin{equation}
	r_{XY}(n_1,n_2) = E[X(n_1) \cdot Y(n_2)]
\end{equation}

\subsubsection{Gemeinsame Statonarit�t (joint stationary)}
\begin{equation}
	r_{XY} = r_{XY}(n_1-n_2) = r_{XY}(\kappa) \quad\text{mit}\quad \kappa=n_1-n_2
\end{equation}
\subsubsection{Eigenschaften der Cross-SOMF}
\begin{subequations}
\begin{align}
r_{XY}(-\kappa) &= r_{YX}(\kappa) \\
|r_{XY}(\kappa)| &\leq \sqrt{r_{XX}(0) \cdot r_{YY}(0)} \\
|r_{XY}(\kappa)| &\leq \frac{1}{2}(r_{XX}(0)+r_{YY}(0))
\end{align}
\end{subequations}
\subsubsection{Unkorreliertheit (uncorrelated)}
\begin{equation}
	r_{XY}(\kappa)=\mu_x \cdot \mu_y
\end{equation}
\subsubsection{Orthogonalit�t}
\begin{equation}
	r_{XY}(\kappa)=0
\end{equation}
\end{document}