%%This is a very basic article template.
%%There is just one section and two subsections.

\documentclass[accentcolor=tud9c,11pt,nochapname]{tudreport}
\usepackage{mathtools}
\usepackage[pdftex,bookmarks=true]{hyperref}
\usepackage{ngerman}
\DeclareMathSizes{11}{11}{8}{8}
\begin{document}

\def\Var{{\rm Var}\,}
\def\E{{\rm E}\,}


\title{Stochastische Signale und Systeme}
\subtitle{Zusammenfassung Formeln}
\subsubtitle{Autor: Daniel Thiem - studium@daniel-thiem.de}

\maketitle




\tableofcontents


\numberwithin{equation}{chapter}
\section*{Vorwort}
Fehler und Verbesserungen bitte an studium@daniel-thiem.de senden.
\chapter{Kombinatorik \& reine Stochastik}

\section{Wahrscheinlichkeitsdichtefunktion}
Sei $F_X(x)$ die Verteilungsfunktion der Zufallsvariablen $X$
\begin{equation}
	f(x) = \frac{dF_X(x)}{dx}
\end{equation}

\subsection{Eigenschaften der Wahrscheinlichkeitsdichtefunktion}
\begin{subequations}
\begin{align}
	f_X(x)&\geq 0 \\
	f_X(x) &= P(X=x)
\end{align}
\end{subequations}

\subsection{Berechnung bei Abh�ngigkeit zu anderer Zufallsvariablen}
Sei $Y=g(X)$ und die Wahrscheinlichkeitsdichtefunktion von Y, $f_y(t)$, sei gesucht, w�hrend die Wahrscheinlichkeitsdichtefunktion $f_x(t)$ gegeben ist,
\begin{equation}
f_y(t)=f_x(g^{-1}(t))\left|\frac{d}{dy}g^{-1}\right|
\end{equation}

\section{Verteilungsfunktion}
$f(t)$ sei die Wahrscheinlichkeitsdichtefunktion der Zufallsvariablen $X$
\begin{equation}
	F(x) = P(X\leq x) = \int\limits_{-\infty}^x f(t)dt 
\end{equation}
\subsection{Eigenschaften der Verteilungsfunktion}
\begin{subequations}
\begin{align}
0 \leq F_X(x) &\leq 1 \\
F_X(\infty)&=1\\
F_X(-\infty)&=0 \\
F_X(x) \text{ist rechtsstetig, d.h. } \notag\\
\lim_{\epsilon \rightarrow 0} F_X(x+\epsilon) &= F_X(x)
\end{align}
\end{subequations}
\subsection{Wahrscheinlichkeitsrechnung mittels der Verteilungsfunktion}
\begin{subequations}
\begin{align}
F(a-) &= \lim_{\epsilon \rightarrow 0}  F_X(x-\epsilon)\\
P(X=a) &= F(a)-F(a-) \\
P(a<X\leq b) &=F(b)-F(a)\\
P(a\leq X<b) &=F(b-)-F(a-)\\
P(a\leq X\leq b) &=F(b)-F(a-)\\
P(X>a)&=1-F(a)
\end{align}
\end{subequations}
\section{Verteilungen}


\subsection{Normalverteilung}
\begin{equation}
f(x) = \frac{1}{\sigma \sqrt{2 \pi}} e^{- \frac{1}{2} \left(\frac{t-\mu}{\sigma}\right)^2}
\end{equation}

\subsection{Rechteckverteilung}
\begin{align}
f(t)&=\begin{cases}
\frac{1}{b-a} & a<t<b \\
0 & \text{sonst}
\end{cases} \\
F(x) &= \begin{cases}
0 & x \leq a \\
\frac{x-a}{b-a} & x \in (a,b] \\
1 & x > b
\end{cases}
\end{align}

\subsection{Exponentialverteilung}
\begin{align}
f(t)&=\begin{cases}
0 & t<0 \\
\lambda e^{-\lambda t} & t\geq 0
\end{cases} \\
F(x) &= \begin{cases}
0 & x < 0 \\
1-e^{-\lambda t} & x\geq 0
\end{cases}
\end{align}

\section{Formel von Bayes}

\begin{equation}
	P(A|B) = \frac{P(A \cap B)}{P(B)} 
	\Rightarrow P(A_k|B) = \frac{P(A_k \cdot P(B|A_k))}{\sum\limits_{i=1}^n P(B|A_i) \cdot P(A_i)}
\end{equation}



\section{Erwartungswerte}
\subsection{Erwartungswertberechnung}
\subsubsection{Allgemein}

Sei $f(x)$ die Wahrscheinlichkeitsdichtefunktion von $X$

\begin{equation}
	\E(X) = \int\limits_{-\infty}^\infty x \cdot f(x) dx
\end{equation}

\subsubsection{Erweitert}

Sei $Y=g(X)$ und $f(x)$ die Wahrscheinlichkeitsdichtefunktion von $X$

\begin{equation}
	\E[Y] = \E[g(X)] =  \int\limits_{-\infty}^\infty g(x) \cdot f(x) dx
\end{equation}

\subsection{Rechenregeln f�r Erwartungswerte}
Sei $A$ eine von $B$ unabh�ngige Zufallsvariable
\begin{equation}
	\E[A \cdot B] = \E[A] \cdot \E[B]
\end{equation}
Sei $X$ eine Zufallsvariable und $a,b$ jeweils Konstanten
\begin{equation}
	\E[aX +b] = a\E[X] + b
\end{equation}
Seien $X_i$ Zufallsvariablen
\begin{equation}
\E\left[\sum\limits_{i=0}^n X_i\right]=\sum\limits_{i=0}^n  \E[X_i]
\end{equation}
\section{Varianz}
\subsection{Berechnung der Varianz}
\begin{equation}
\Var(X)=\E(X^2)-\E(X)^2
\end{equation}
\subsection{Rechenregeln f�r Varianzen}
\begin{equation}
\Var(aX+b)=a^2 \Var(x)
\end{equation}
Seien $X_i$ Zufallsvariablen
\begin{equation}
\Var\left[\sum\limits_{i=0}^n X_i\right]=\sum\limits_{i=0}^n  \Var[X_i]
\end{equation}
\chapter{Discrete-Time-Fourier-Transformation}

\section{Abtastung}
\subsection{Im Zeitbereich}
Sei $x_c(t)$ das zu abtastende Signal und $T_s=\frac{1}{f_s}$ die Abtastdauer bzw. Abtastfrequenz
\begin{equation}
x_s(t)=\sum\limits_{n=-\infty}^\infty x_c(nT_s)\delta(t-nT_s)
\end{equation}
\subsection{Im Frequenzbereich}

\begin{align}
X_s(j\Omega)&=\frac{1}{T_s}\sum\limits_{k=-\infty}^\infty X_c(j(\Omega-\frac{2\pi k}{T_s})) \notag\\
	&=\frac{1}{T_s}\sum\limits_{k=-\infty}^\infty X_c(j\Omega-kj\Omega_s) \quad \text{mit} \quad \Omega_s=\frac{2\pi}{T_s}\\
\end{align}
\section{Transformation}
\subsection{R�cktransformation}
\begin{align}
x[n]&=\int\limits_{-\pi}^\pi X(e^{j\omega})e^{j\omega n}d\omega 
\end{align}
\subsection{Zusammenhang $\Omega$ und $n$} \label{omegaToDescrete}
ACHTUNG: Dieser zusammenhang ist in SSS etwas anders im gegensatz zu dem Hilfsblatt von DSS
\begin{equation}
\omega = \Omega T_s
\end{equation}
\subsection{Berechnen einer �bertragungsfunktion im zeitdiskreten Fall}
\begin{enumerate}
	\item Zeitkontinuierliches $H(e^{j\omega})=\frac{Y(e^{j\omega})}{X(e^{j\omega})}$ berechnen
	\item Formel aus (\ref{omegaToDescrete}) einsetzen, um $H(j\Omega)$ zu erreichen
\end{enumerate}
\chapter{Prozesse}

\section{Strikte Stationarit�t} \label{stationarystrict}
\begin{equation}
F_x(x_1,\dots ,x_N;n_1,\dots,n_N) = F_x(x_1,\dots ,x_N;n_1+n_0,\dots ,n_N+n_0) \quad \text{mit $N\rightarrow \infty$}
\end{equation}

\section{Second order moment function(SOMF)} \label{somf}
\begin{equation}
r_{XX}(n_1,n_2)=\E[X(n_1)X(n_2)]
\end{equation}
\subsection{Station�r im weiteren Sinne} \label{stationary}
\begin{subequations}
\begin{align}
\E[X(n)]&=\text{const.} \\
r_{XX}(n_1,n_2) &= r_{XX}(\kappa) = \E[X(n+\kappa)\cdot X(n)] \quad \text{mit} \quad \kappa = |n_2-n_1|
\end{align}
\end{subequations}
\subsection{Eigenschaften der SOMF}
\begin{subequations}
\begin{align}
r_{XX}(0) &= \E[X(n)^2]=\sigma_X^2+\mu_x^2 \\
r_{XX}(\kappa) &= r_{XX}(-\kappa) \\
r_{XX}(0) &\geq|r_{XX}(\kappa)| \quad ,|\kappa|>0
\end{align}
\end{subequations}

\section{Cross-SOMF}
\begin{equation}
	r_{XY}(n_1,n_2) = \E[X(n_1) \cdot Y(n_2)]
\end{equation}

\subsection{Gemeinsame Statonarit�t (joint stationary)}\label{jointstationary}
\begin{equation} 
	r_{XY} = r_{XY}(n_1-n_2) = r_{XY}(\kappa) \quad\text{mit}\quad \kappa=n_1-n_2
\end{equation}
\subsection{Eigenschaften der Cross-SOMF}
\begin{subequations}
\begin{align}
r_{XY}(-\kappa) &= r_{YX}(\kappa) \\
|r_{XY}(\kappa)| &\leq \sqrt{r_{XX}(0) \cdot r_{YY}(0)} \\
|r_{XY}(\kappa)| &\leq \frac{1}{2}(r_{XX}(0)+r_{YY}(0))
\end{align}
\end{subequations}
\subsection{Unkorreliertheit (uncorrelated) anhand der Cross-SOMF}
\begin{equation}
	r_{XY}(\kappa)=\mu_x \cdot \mu_y = \E[X(n+\kappa)]\E[Y(n)]
\end{equation}
\subsection{Orthogonalit�t} \label{ortho}
\begin{equation}
	r_{XY}(\kappa)=0
\end{equation}
\section{Kovarianz (Covariance,Central-SOMF)}
\begin{subequations}
\begin{align}
c_{XX}(n+\kappa,n) &= \E[(X(n+\kappa)-\E[X(n+\kappa)]) \cdot (X(n)-\E[X(n)])] \\
c_{XX}(n+\kappa,n) &= r_{XX}(n+\kappa,n)-\E[X(n+k)]\E[X(n)]
\end{align}
\end{subequations}
\subsection{Eigenschaften der Kovarianz}
Falls $X$ zumindest \emph{station�r im weiteren Sinne}(\ref{stationary}) ist, gilt
\begin{equation}
c_{XX}(\kappa)=r_{XX}(\kappa)-(\E[X(n)])^2 
\end{equation}

\subsection{�berf�hrung der Central-SOMF in die Varianz}
\begin{equation}
c_{XX}(0)=\Var(X)
\end{equation}
\section{Kreuz-Kovarianz (Cross-covariance)}
\begin{subequations}
\begin{align}
c_{XY}(n+\kappa,n) &= \E[(X(n+\kappa)-\E[X(n+\kappa)]) \cdot (Y(n)-\E[Y(n)])] \\
c_{XY}(n+\kappa,n) &= r_{XY}(n+\kappa,n)-\E[X(n+k)]\E[Y(n)]
\end{align}
\end{subequations}

\subsection{Eigenschaften der Kreuzkovarianz}

Falls $X$ und $Y$ zumindest \emph{gemeinsam station�r im weiteren Sinne }(\ref{jointstationary}) sind, gilt:

\begin{equation}
c_{XY}(\kappa)=r_{XY}(\kappa)-\E[X(n)]\E[Y(n)]
\end{equation}
\subsection{Unkorreliertheit (uncorrelated) anhand der Kreuzkovarianz}\label{uncorrelated}
\begin{equation}
	c_{XY}(\kappa)=0
\end{equation}

\section{Komplexe Prozesse}
Seien $X(n)$ und $Y(n)$ reale Zufallsprozesse, so ist
\begin{equation}
Z(n)\hat{=}  X(n)+jY(n)
\end{equation}
ein Komplexer Zufallsprozess
\subsection{Erwartungswert eines Komplexen Zufallsprozess}
\begin{equation}
\E[Z(n)]=\E[X(n)]+j\E[Y(n)]
\end{equation}
\subsection{SOMF eines Komplexen Zufallsprozess}
\begin{equation}
r_{ZZ}(n_1,n_2)=\E[Z(n_1) \cdot Z(n_2)^*]
\end{equation}
\subsubsection{Besondere Eigenschaften}
F�r einen komplexen Zufallsprozess, welcher \emph{station�r im weiteren Sinne}(\ref{stationary}) ist, gilt
\begin{equation}
r_{ZZ}(-\kappa)=r_{ZZ}(\kappa)^*
\end{equation}
\subsection{cross-SOMF komplexer Zufallsprozesse}

\begin{equation}
	r_{Z_1Z_2}(n_1,n_2) = \E[Z_1(n_1) \cdot Z_2(n_2)^*]
\end{equation}

\subsection{Kovarianz (Covariance) eines komplexen Zufallsprozess}
\begin{equation}
c_{ZZ}(n+\kappa,n) = \E[(Z(n+\kappa)-\E[Z(n+\kappa)]) \cdot (Z(n)-\E[Z(n)])^*]
\end{equation}

\subsection{Kreuzkovarianz(cross-covariance) komplexer Zufallsprozesse}
\begin{equation}
c_{Z_1Z_2}(n+\kappa,n) = \E[(Z_1(n+\kappa)-\E[Z_1(n+\kappa)]) \cdot (Z_2(n)-\E[Z_2(n)])^*] 
\end{equation}

\subsection{Eigenschaften komplexer Zufallsprozesse}
\emph{Unkorreliertheit} verh�lt sich wie (\ref{uncorrelated}), genauso wie \emph{Orthogonalit�t} (\ref{ortho})
\chapter{Spektraldichten (Power Spectral Density)}
% 
% 
% \section{Normierte Energie in einem Zeitintervall}
% \begin{subequations}
% \begin{align}
% E_N&=\sum\limits_{n=-M}^M x_N(n)^2 \\
% &=\int\limits_{-\pi}^\pi \left|X_N\left(e^{j\omega}\right)\right|^2 \frac{d\omega}{2\pi}
% \end{align}
% \end{subequations}
% \section{Durchschnittliche Leistung in einem Zeitintervall}
% \begin{subequations}
% \begin{align}
% P_N&=\frac{1}{2M+1}\sum\limits_{n=-M}^M x_N(n)^2 \\
% &=\int\limits_{-\pi}^\pi \frac{\left|X_N\left(e^{j\omega}\right)\right|^2}{2M+1} \frac{d\omega}{2\pi}
% \end{align}
% \end{subequations}

\section{Leistungsdichte}
\subsection{Leistungsspektraldichte (Power Spectral Density,PSD)} \label{psd}
\begin{align}
S_{XX}(e^{j\omega},\xi) &= \lim_{M \rightarrow \infty} \frac{\E\left[\left|X_N\left(e^{j\omega},\xi\right)\right|^2\right]}{2M+1}\\
\text{mit}\notag\\
X_N(e^{j\omega},\xi) &=\sum\limits_{n=-M}^M x_N(n,\xi) e^{-j\omega n}
\end{align}
\subsubsection{Eigenschaften der Leistungsspektraldichte}
\begin{subequations}
\begin{alignat}{3}
S_{XX}(e^{j \omega})^*&=S_{XX}(e^{j \omega}) \quad &\text{mit}\quad X(n)\in \mathbb{C} \\
S_{XX}(e^{j \omega})&\geq 0 \quad &\text{mit}\quad X(n)\in \mathbb{C} \\
S_{XX}(e^{-j \omega})&=S_{XX}(e^{j \omega}) \quad &\text{mit}\quad X(n)\in \mathbb{R} 
\end{alignat}
\end{subequations}
\subsection{Durchschnittliche Leistung eines Zufallsprozesses}
\begin{subequations}
\begin{align}
P_{XX}&=\int\limits_{-\pi}^\pi S_{XX}(e^{j\omega}) \frac{d\omega}{2\pi} = r_{XX} (0)\\
&=\lim_{M \rightarrow \infty}\int\limits_{-\pi}^\pi \frac{\E\left[\left|X_N\left(e^{j\omega},\xi\right)\right|^2\right]}{2M+1} \frac{d\omega}{2\pi}
\end{align}
\end{subequations}
\subsection{Kreuzleistungsdichte (cross-power density)}
\begin{equation}
S_{XY}(e^{j\omega},\xi) = \lim_{M \rightarrow \infty} \frac{\E\left[X_N\left(e^{j\omega},\xi\right)Y_N\left(e^{j\omega},\xi\right)^*\right]}{2M+1}\\
\end{equation}
\subsubsection{Eigenschaften der Kreuzleistungsdichte}
\begin{subequations}
\begin{alignat}{3}
S_{XY}(e^{j \omega})^*&=S_{YX}(e^{j \omega}) \quad &\text{mit}\quad X(n),Y(n)\in \mathbb{C} \\
S_{XY}(e^{j \omega})^*&=S_{YX}(-e^{j \omega}) \quad &\text{mit}\quad X(n),Y(n)\in \mathbb{R} \\
\mathfrak{Re}\{S_{XY}(e^{j \omega})\}&\text{ und }\mathfrak{Re}\{S_{YX}(e^{j \omega})\} &\text{sind gerade, wenn } X(n),Y(n) \in \mathbb{R}\\
\mathfrak{Im}\{S_{XY}(e^{j \omega})\}&\text{ und }\mathfrak{Im}\{S_{YX}(e^{j \omega})\} &\text{sind ungerade, wenn } X(n),Y(n) \in \mathbb{R}\\
S_{XY}(e^{j \omega})&=S_{YX}(e^{j \omega}) =0 \quad &\text{wenn $X(n)$ und $Y(n)$ orthogonal (\ref{ortho})} 
\end{alignat}
\end{subequations}
\subsection{Durchschnittliche Kreuzleistung zweier Zufallsprozesse}
\begin{equation}
P_{XY}=\int\limits_{-\pi}^\pi S_{XY}(e^{j\omega}) \frac{d\omega}{2\pi}
\end{equation}

\subsection{Wiener-Khinchine theorem}
Ist $X(n)$ ein \emph{im weiteren Sinne station�rer}(\ref{stationary}) Zufallsprozess, do kann die \emph{Leistungsspektraldichte} (\ref{psd}) 
aus der Fourier-Transformation der \emph{Momentenfunktion zweiter Ordnung(SOMF)} (\ref{somf}) gewonnen werden:
\begin{subequations}
\begin{align}
S_{XX}(e^{j\omega})&=\mathcal{F}\{r_{XX}(\kappa)\}=\sum\limits_{k=-\infty}^\infty r_{XX}(\kappa) e^{-k\omega \kappa} \\
&\quad\text{und invers} \notag\\
r_{XX}(\kappa)&=\mathcal{F}^{-1}\{S_{XX}(e^{j\omega})\}=\int\limits_{-\pi}^\pi S_{XY}(e^{j\omega \kappa})\frac{d\omega}{2\pi}
\end{align}
\end{subequations}

\subsection{Kreuzleistungsdichte durch Cross-SOMF}
\begin{equation}
S_{XY}(e^{j\omega})=\mathcal{F}\{r_{XY}(\kappa)\}=\sum\limits_{k=-\infty}^\infty r_{XY}(\kappa) e^{-k\omega \kappa}
\end{equation}


\section{Koh�renz (coherence)} \label{coherence}
\begin{equation}
	\rm{Coh}_{XY}(e^{j\omega})=\frac{\left|S_{XY}(e^{j\omega})\right|^2}{S_{XX}(e^{j\omega})S_{YY}(e^{j\omega})}
\end{equation}
\subsection{Eigenschaften der Koh�renz}
Die Koh�renz zwischen den Zufallsprozessen $X(n)$ und $Y(n)$ besagt, wie gut $X$ zu $Y$ bei einer gegebenen Frequenz $\omega$ korrespondiert.
\begin{equation}
0\leq\rm{Coh}_{XY}(e^{j\omega})\leq 1
\end{equation}



\chapter{Sonstiges}
\section{Spezielle Funktionen}
\subsection{Gaussian white noise process}
Gau�sches wei�es Rauschen ist immer \emph{station�r} (\ref{stationarystrict}) 
\begin{subequations}
\begin{align}
r_{WW}(\kappa)&=\sigma_W^2\delta(\kappa) \\
S_{WW}(e^{j \omega})&=\sigma_W^2
\end{align}
\end{subequations}

\subsection{Kronecker delta function}
\begin{equation}
\delta(\kappa)=\begin{cases}
1 &\kappa = 0\\
0 &\kappa \neq 0
\end{cases}
\end{equation}
\end{document}
